\section{INTRODUCTION}
    \subsection{Background of the project}
            The Ivatan language and folklore are integral components of the cultural identity and heritage of the Ivatan people in the northernmost province of Batanes, Philippines. These intangible cultural elements—ranging from oral traditions, proverbs, and stories to unique lexical structures—are essential carriers of ancestral knowledge, values, and worldviews. However, modernization, the growing dominance of mainstream languages like Filipino and English, and the widening generational gap in cultural transmission have placed Ivatan language and folklore at serious risk of extinction.

            Globally, indigenous languages and oral traditions are disappearing at an alarming rate. According to UNESCO, nearly half of the world’s approximately 6,000 languages are endangered, with one language disappearing every two weeks (Evans, 2009). With each lost language, a rich repository of indigenous wisdom, history, and identity disappears as well. In the case of the Ivatan community, younger generations are becoming less fluent due to limited exposure and use of the language in schools, media, and digital spaces. As fewer native speakers remain, the intergenerational transmission of the Ivatan language and folklore continues to decline. Without proactive intervention, vital aspects of Ivatan culture, including traditional phrases, proverbs, and folklore, face the threat of being permanently lost.
            
            The problem is further compounded by the lack of accessible, participatory, and technologically driven platforms for cultural preservation. While there have been commendable efforts to document Ivatan language and traditions through printed dictionaries, academic theses, and ethnographic research, these resources remain fragmented, static, and difficult to access—particularly for non-academic users and the youth. Existing solutions are largely limited to traditional media, which do not leverage modern information and communication technologies to foster deeper engagement or community participation. The absence of such a system hinders the effective transmission of cultural knowledge and prevents interested learners from easily accessing and engaging with linguistic and folklore resources.
            
            Recognizing this challenge, this capstone project aims for the development of Chirin Ivatan: A Community-Based Information System for Language and Folklore Conservation. This web-based application is envisioned as an innovative digital solution that aims to preserve and revitalize the Ivatan language. This system will provide a digital dictionary, folklore archive, and community-driven content submission features. Unlike previous efforts, this solution will harness modern web technologies to create an accessible, user-friendly, and sustainable platform that encourages participation from native speakers, educators, researchers, and the younger Ivatan population. The system aims to ensure that the transmission of linguistic and cultural knowledge continues in an engaging and meaningful way.
            
            The proposed system will consist of three core components designed to address both preservation and engagement goals. First, it will include a comprehensive digital English-Ivatan dictionary enriched with audio pronunciation capabilities to aid in the accurate learning and revitalization of the language. Second, an interactive folklore archive will serve as a digital repository for Ivatan traditions, legends, proverbs, and oral stories, making them easily accessible and systematically organized for exploration and study. Third, the system will feature a community-driven content submission mechanism, enabling verified contributors—particularly elders, educators, and cultural bearers—to actively participate in the documentation and validation of language and folklore materials. To further promote user engagement and educational value, the platform will incorporate gamification elements such as quizzes and progress tracking, encouraging continuous interaction and learning. 
            
            The proposed system will be developed using React.js for the frontend and Django with Python for the backend. PostgreSQL will serve as the main database to handle structured data like dictionary entries, folklore content, and user profiles. To enhance user experience, the system will feature audio pronunciation via the Web Audio API and include gamified elements like quizzes and progress tracking to encourage learning. User-centered design principles will be applied to ensure accessibility and ease of use. Partnerships with cultural organizations, local educators, and Ivatan community members will also be central to the content validation and sustainability of the project. The final outcome is a functional, community-driven web platform.

            The ultimate goal of this project is to create a functional, sustainable, and inclusive digital platform for preserving the Ivatan language and folklore. It aims to deliver a working web application that is accessible to users of all ages and technical backgrounds, encouraging greater community participation in cultural preservation. The platform is expected to raise awareness and support learning, especially among younger generations, while offering a model that can be adapted for other endangered languages and communities. In essence, Chirin Ivatan is more than just a digital tool—it is a cultural preservation effort grounded in community collaboration and powered by technology. By connecting tradition with innovation, the project seeks to keep the Ivatan people's rich linguistic and folkloric heritage alive in the digital age.


    \subsection{Objectives}
        To design and develop a community-based web application that facilitates the preservation and promotion of the Ivatan language and folklore through digital tools that enable documentation, learning, and community participation.
            \begin{itemize}
                \item To develop a user-friendly English-Ivatan digital dictionary with audio pronunciation features using Python and the Web Audio API
                \item To implement a folklore documentation module that archives at least 50 Ivatan stories, myths, and proverbs in text and audio formats, enabling browsing, submission, and community validation 
                \item To integrate interactive learning tools—such as quizzes, progress tracking, and gamified features—to improve language engagement and retention
                \item To build a contribution and moderation system that allows verified cultural stakeholders to submit and review content, with at least five active reviewers testing the system by the end of the development phase.
                \item To ensure the platform’s accessibility and usability through user-centered design, targeting users across varying levels of digital literacy and collecting feedback from at least 20 test users.
                \item To release the system as an open-source project by April 2026, providing documentation that allow other ethnolinguistic communities to replicate and adapt the platform for their own language preservation efforts.
            \end{itemize}
            
    \subsection{Scope and Limitations}

        \subsubsection{Scope}
            The project’s primary scope includes the design, development, and partial deployment of a web-based application featuring three main modules: a community contribution system, an Ivatan-English digital dictionary with audio support, and a folklore documentation archive. Interactive learning tools, such as quizzes and gamified features, are also planned but will be explored in later stages of the project, depending on development progress and available programming capacity.
            
            This study will specifically cover the following:
            \begin{itemize}
                \item User Interface Design using React.js to ensure accessibility and ease of use for a wide range of users, including Ivatan elders, students, educators, and cultural researchers.
                \item Back-End Development using Django, Python, and PostgreSQL
                \item Audio Integration through the Web Audio API to provide pronunciation guides for Ivatan words, thereby supporting auditory language learning.
                \item Community Participation Tools, including submission forms, moderation workflows, and a gamification system that promotes user engagement through quizzes and leaderboards.
                \item Collaboration with Local Stakeholders, such as schools and non-profit organizations to ensure cultural relevance and encourage community involvement.
            \end{itemize}
            By clearly defining these boundaries, the project remains focused on building a functional prototype that demonstrates the potential of information systems in addressing cultural preservation challenges. It does not extend into full nationwide deployment, mobile app development, or real-time collaboration tools, which may be considered in future iterations.
        \subsubsection{Limitations}
            Despite the project's ambition to digitally preserve the Ivatan language and folklore, several limitations, primarily due to technical, logistical, and contextual constraints, are acknowledged. 

            First, the system's initial content will depend heavily on the availability and willingness of local contributors, such as elders and educators, to share and validate linguistic and folkloric data. This could result in limited entries during the early stages of development.

            Second, due to the project proponent’s current beginner-level programming skills, the system may initially rely on simplified implementations or external frameworks like Firebase before transitioning to Django with PostgreSQL. This learning curve may impact the development timeline and feature completeness.

            Third, the platform’s reach and testing will be limited primarily to users with internet access and basic digital literacy, potentially excluding some members of the Ivatan community. Due to geographical isolation and internet connectivity issues in Batanes, real-time collaboration with native speakers and local contributors may be limited. The system is designed with asynchronous contributions in mind, but the richness of content depends heavily on user input. Furthermore, audio features and real-time interactions depend on users’ device compatibility and bandwidth, which may vary across locations. 

            Lastly, cultural validation is subject to the availability of experts and community reviewers, which may affect the speed and consistency of content moderation.

            These limitations are recognized to maintain realistic expectations and will be addressed progressively through partnerships, iterative updates, and future developments that may be beyond the scope of this capstone.
