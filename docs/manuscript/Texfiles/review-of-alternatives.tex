\section{REVIEW OF EXISTING ALTERNATIVES}

                Since cultural heritage is gradually being lost due to globalization and the digital divide, maintaining indigenous languages and folklore is a global concern. This is also true for the Ivatan language and folklore, which are endangered by gaps in generational knowledge and a dearth of easily accessible and cooperative platforms for preservation. Oral transmission, written dictionaries, and ethnographic publications have historically been the main methods used to preserve the Ivatan language and traditions. Despite their value, these approaches are not sustainable because of urban migration, modernization, and language loss between generations. There are not many thorough and detailed resources available to younger Ivatans, particularly those who were raised in cities or outside of Batanes, that would enable them to connect with their linguistic and cultural history. Although there have been various attempts to document and archive Ivatan cultural heritage, these initiatives remain fragmented and limited in their accessibility. This chapter explores existing alternatives, evaluates their effectiveness, and highlights the unique contributions of the Chirin Ivatan.
                
                Several strategies have been implemented to preserve indigenous languages and folklore, ranging from traditional documentation methods to digital initiatives. 

                \subsection {Printed and Academic Documentation}
                The Ivatan language together with other Philippine ethnolinguistic languages has received attention through printed dictionaries  and linguistic study as well as folklore compilations. The general public lacks access to these resources because they  are scarce and expensive to produce and written in academic language while being essential for academic research and cultural preservation.

                The Ivatan language has received extensive documentation especially regarding its vocabulary and grammatical structure. The  Ivatan-Filipino-English Dictionary by Cesar A. Hidalgo stands as one of the  most recognized works after its publication by the Academics Foundation in March 1998. The comprehensive reference  functions as a primary academic resource while delivering valuable linguistic information. The printed format of this resource restricts  its accessibility and user engagement because younger generations prefer digital interactive platforms which characterize many traditional language products.

                The  Ivatan Language Packets also represent a vital resource that the public can access since their publication in  1993. The resources exist to provide Peace Corps volunteers with language instruction when they serve in Batanes.  The resources contain essential information about language communication and cultural traditions. These packets existed for practical language learning purposes  but lacked the intention to maintain their use beyond short-term educational needs. The resources remain unavailable for modern  educational purposes because they have not received updates or digital transformations.

                Ethnographic studies have also focused on Ivatan oral traditions, stories, and customs, contributing to the growing body of information that is still preserved in academic circles. However, only researchers have access to these works since they are often housed in university libraries or research archives; community members, especially younger generations, who would benefit from a more approachable and interesting manner to learn about their past, are left out.

                \subsection {Community-Driven Language Revitalization Programs}
                Various schools and cultural centers, particularly the National Commission on Indigenous People, have initiated community-based programs to encourage younger generations to learn their native languages through traditional storytelling, songs, and cultural activities. These efforts recognize the importance of early exposure to indigenous languages in maintaining cultural identity.

                However, the removal of the Mother Tongue-Based Multilingual Education (MTB-MLE) subject from the primary education curriculum poses a significant setback to these revitalization efforts. Without structured language instruction in schools, it becomes more uncertain whether younger generations can retain fluency in their native tongue.

                Additionally, Ivatan communities have long relied on elders as cultural bearers, transmitting knowledge and traditions through oral storytelling. While this intergenerational approach remains valuable, it is increasingly challenged by the dwindling number of fluent native speakers and the growing preference among youth for digital forms of content and learning. These limitations highlight the urgent need for digital tools that can complement traditional methods and adapt to contemporary learning preferences.

                \subsection {Existing Digital Initiatives}
                Technological developments have made it possible for digital platforms dedicated to protecting endangered languages and cultural assets to arise. Data on endangered languages, including vocabulary, grammar, and pronunciation, can be accessed online thanks to initiatives like the Endangered Languages Project and digital repositories like FirstVoices and Living Dictionaries. The potential of digital involvement in language retention has been demonstrated by the incorporation of indigenous languages into gamified learning methods by apps such as Duolingo.

                However, while these initiatives demonstrate the power of digital preservation, most are generalized and do not address the unique context of the Ivatan language or its accompanying oral traditions. Moreover, many global platforms are limited to linguistic documentation and lack integrated folklore modules or mechanisms for community-driven contributions which are also key elements for holistic cultural preservation.

                In the Philippine context, Project Marayum stands out as a significant initiative for indigenous language preservation. Spearheaded by the University of the Philippines, the project focuses on developing web-based dictionaries for endangered Philippine languages, providing an accessible digital platform for communities to document and share their native tongues. Its collaborative model allows local language experts, educators, and community members to contribute directly to the creation of these resources, empowering grassroots participation in linguistic preservation.

                Project Marayum served as an inspiration for the development of Chirin Ivatan, especially in its vision of combining technology with community-driven content. While the platform already includes some Northern Luzon languages such as Itneg and Kankanaey, Ivatan has yet to be formally represented. This highlights a clear opportunity for Chirin Ivatan to fill a critical gap by dedicating a focused platform to the Ivatan language and folklore, expanding on Marayum’s foundational approach with added features such as folklore documentation, audio pronunciation, and interactive learning tools.

                Several initiatives have been developed to address the needs of indigenous language preservation documenting ivatan terms:
                    \begin{itemize}
                        \item SIL Philippines: Ibatan-English Dictionary- SIL Philippines has played a crucial role in documenting the languages of minority communities across the archipelago, including the Ivatan and Ibatan languages spoken in the Batanes and Babuyan islands. One of their most significant contributions is the Ibatan-English Dictionary, a linguistically grounded compilation of Ibatan vocabulary, developed through extensive fieldwork and collaboration with native speakers. The dictionary includes 5,602 headwords, with an English index containing 9,109 entries, a Filipino index with 2,556 entries, an Ilokano index with 3,471 entries, and an Ivatan index with 1,703 entries—offering a multilingual perspective on Ibatan semantics. However, the dictionary remains static and text-heavy, primarily designed for linguistic research rather than public engagement. 
                        \item Glosbe: Dictionary English – Ivatan- Glosbe is an open, collaborative multilingual dictionary platform that includes an English–Ivatan section. While it currently contains only a limited number of Ivatan terms and example phrases, it reflects initial efforts to digitize Ivatan vocabulary in an accessible online environment. However, the platform lacks pronunciation support, grammar explanations, and community engagement features, limiting its utility for language learners or cultural advocates.
                        \item Ivatan Language Wikipedia Page- The Ivatan language page on Wikipedia offers a brief overview of the language’s structure, including its phonology, basic grammar rules, and pronoun system. This page serves as a useful introduction to Ivatan for linguists and language enthusiasts, and can act as a starting point for deeper research. However, it remains a basic reference that lacks comprehensive vocabulary, context-rich usage examples, and multimedia support such as audio clips or oral storytelling.
                    \end{itemize}

                \subsection {How Chirin Ivatan Stands Out}
                The global and local landscape of digital language preservation has introduced a variety of platforms and tools that offer important lessons for designing effective, culturally responsive systems. However, despite these strengths, existing solutions often fall short in critical areas. 

                This is where Chirin Ivatan emerges as a transformative alternative. Unlike other models that focus solely on linguistic documentation or static dictionary creation, Chirin Ivatan offers a comprehensive, community-driven digital platform that intertwines language preservation with cultural storytelling. The platform enables native Ivatan speakers, educators, and cultural advocates to collaboratively contribute and review content, ensuring authenticity and inclusivity. It features a digital Ivatan-English dictionary enhanced with phonetic transcriptions and a dedicated folklore module will document myths, legends, proverbs, and songs—helping to preserve not just the language, but also the values, worldview, and identity embedded in Ivatan oral traditions. The platform is planned to incorporate interactive and gamified learning elements such as quizzes and leaderboards. While these features will be implemented progressively, they are an important component in attracting and retaining users, especially among younger generations. 

                In conclusion, whereas current digital projects have established crucial foundations for language preservation, their reach, interactivity, and cultural depth are frequently constrained. By combining several features—language documentation, folklore preservation, and interactive learning—into a single user-friendly platform, Chirin Ivatan overcomes these constraints. It adapts its methodology to the unique requirements of the Ivatan community while building on the advantages of earlier models. Chirin Ivatan hopes to leave a lasting impact on the revival and preservation of Ivatan culture and language.
